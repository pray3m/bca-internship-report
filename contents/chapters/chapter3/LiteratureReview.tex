\section{Literature Review}
A number of studies and projects have examined the benefits and challenges of adopting cloud-based systems in healthcare. For example, research by Hsu et al. demonstrated that moving to a SAAS-based management system can significantly reduce operational costs and improve data accuracy. Other studies have focused on the scalability of cloud solutions, noting that multi-tenancy and high availability are critical for systems that serve large, distributed organizations.

Several key findings from the literature include:
\begin{itemize}
  \item SAAS-based hospital management systems can lead to increased efficiency by automating routine processes such as patient registration, appointment scheduling, and billing.
  \item Cloud-based models offer significant cost benefits due to reduced infrastructure requirements and simplified maintenance.
  \item The implementation of SAAS models comes with challenges, particularly in ensuring data privacy and meeting regulatory standards in healthcare.
  \item Previous projects in similar domains have successfully deployed cloud-based systems that improved inter-departmental communication and allowed real-time updates of patient records.
\end{itemize}

Overall, the literature supports the shift towards SAAS models in healthcare. It highlights not only the economic and operational benefits of these systems but also the importance of addressing security, privacy, and compliance issues during development and deployment. The findings from these studies provide a solid foundation for the design and implementation of the SAAS-based HMS at SolveeTech Pvt. Ltd., and they inform the choices made during the project’s development.
